\documentclass[12pt]{article}
\usepackage{fontspec}
\usepackage[utf8]{inputenc}
\setmainfont{Bodoni 72 Book}
\usepackage[paperwidth=9in,paperheight=12in,margin=1in,headheight=0.0in,footskip=0.5in,includehead,includefoot,portrait]{geometry}
\usepackage[absolute]{textpos}
\TPGrid[0.5in, 0.25in]{23}{24}
\parindent=0pt
\parskip=12pt
\usepackage{nopageno}
\usepackage{graphicx}
\graphicspath{ {./images/} }
\usepackage{amsmath}
\usepackage{tikz}
\newcommand*\circled[1]{\tikz[baseline=(char.base)]{
            \node[shape=circle,draw,inner sep=1pt] (char) {#1};}}

\begin{document}

\begin{center}
\huge FOREWORD
\end{center}

\begingroup
\begin{center}
\leftskip0.5in
{\setmainfont{Source Han Serif SC}\selectfont
“ . . .

`一个无解的圈,总会回到开头。'

 

`别哭。'

 

`好的。'

. . . "

\rightskip\leftskip
\phantom{text} \hfill - 南亭湿人}
\end{center}
\endgroup

\begin{center}
\huge NOTES FOR THE INTERPRETERS
\end{center}

\vspace*{2\baselineskip}

\begingroup
\textbf{General: \circled{1} Dynamics} in this score are effort dynamics, representing the physical force behind an action rather than the sounding dynamic. In the case of the celli and percussions, this corresponds to implement pressure. In the case of the bass clarinet, this corresponds to breath pressure. \textbf{\circled{2} Playing techniques} apply only to the note to which they are attached. If a technique is to persist for longer than a single note, a hooked, dashed line will span the music as long as the technique is active. \textbf{\circled{3} Dashed arrows above the staff} indicate a gradual transition from one technique or tempo to another. \textbf{\circled{4} Time signatures whose denominators are not a power of two} are to be understood as a type of metric modulation wherein the pulse shifts to a prolation indicated by the denominator. For example, \textbf{1/6} will contain one ``sixth" note, which is one-sixth of a whole note, or, a triplet quarter note. When these time signatures are active, tuplet brackets which are open on the right side similarly indicate the prolation of a note alone, rather than the number of beats in the prolation. \textbf{\circled{5} Blank measures} are to be understood as full-measure rests. \textbf{\circled{6} Flat glissandi} are sometimes used for the same function as ties. \textbf{\circled{7}} After temporary \textbf{accidentals}, cancellation marks are printed also in the following measure ( for notes in the same octave ) and, in the same measure, for notes in other octaves, but they are printed again if the same note appears later in the same measure, except if the note is immediately repeated. 
\endgroup

\begingroup
\textbf{Bass Clarinet: \circled{1} This score is transposed} so that the written pitch is one major 9th above the sounding pitch. \textbf{\circled{2} Multiphonics} are indicated with a fingering diagram above the fundamental pitch. \textbf{\circled{3} Teeth on reed} is paired with curved lines above the staff which suggest a contour of vibrato. These graphics may be interpreted freely. \textbf{\circled{4} Air sound} is notated on a two-line staff wherein the bottom line indicates fingering the B-flat fundamental and the top line indicates fingering the D-flat fundamental. \textbf{\circled{5} All glissando lines} indicate portamento, as opposed to glissando. \\
\endgroup

\pagebreak

\begingroup
\textbf{Percussions: \circled{1} The percussionist's instruments} are a large lion's roar, a large slit drum, a large frame drum, an ocean drum, a Chinese cymbal, a set of wooden wind chimes, and a set of stone wind chimes. \textbf{\circled{2} The percussionist's implements} are a pair of drum sticks, a pair of soft rubber mallets, a triangle beater, a bow, and two sponges. \textbf{\circled{3} The lion's roar} may be played with a \textbf{pizzicato} or \textbf{pizz.} direction, indicating to pull the string tight with one hand and pluck it with the other. Without this indication, the lion's roar should be played as normal. \textbf{\circled{4} Circular arrow articulations} indicate to draw the implement around the circumference of the instrument within the duration of the articulated note. 
\endgroup

\begingroup
\textbf{Violoncelli: \circled{1} Scordatura:} The \textbf{fourth string} of the \textbf{first cello} should be detuned a major second to \textbf{B-flat}. The \textbf{first string} of the \textbf{first cello} should be detuned to an \textbf{E} at a ratio of \textbf{3/1} of \textbf{A 1}, the pitch of the second cello's fourth string. The \textbf{fourth string} of the \textbf{second cello} should be detuned a minor third to \textbf{A}. The \textbf{first string} of the \textbf{second cello} should be detuned to an \textbf{F} at a ratio of \textbf{3/1} of \textbf{B-flat 1}, the pitch of the first cello's fourth string. The cellists should not play on the fourth or first strings unless directed. When playing on detuned strings, the pitch is transposed to the physical playing position on the string rather than sounding pitch. \textbf{\circled{2} Finger pressure} is indicated with note head shape, wherein a \textbf{round note head} indicates a fully closed or open string, a \textbf{triangular note head} indicates half-pressure, and a \textbf{diamond-shaped note head} indicates harmonic pressure. \textbf{\circled{3} Artificial harmonics} are notated using a white harmonic note head for the touched harmonic over a round note head for the closed string. \textbf{\circled{4} Arrows in the staff} indicate a gradual change from one finger pressure to another. \textbf{\circled{5} Bow angle} is indicated using degree articulations, wherein \textbf{45°} indicates pointing the tip of the bow as far upward as is comfortable, and \textbf{-45°} indicates pointing the tip of the bow as far downward as is comfortable. \textbf{\circled{6} Abbreviations} used in this score are \textbf{trem.} for tremolo, \textbf{pont.} for sul ponticello, \textbf{tast.} for sul tasto,  \textbf{legno bat.} for col legno battuto, \textbf{legno trat.} for col legno tratto, \textbf{flaut.} for flautando, \textbf{scratch} for scratch tone, and \textbf{bridge} for playing directly on the bridge. Abbreviations may be further shortened to their first one, two, or three letters to conserve space, but only if they have appeared in their full form at the beginning of the current phrase. 
\endgroup

\begingroup
\textbf{Microtones: \circled{1} The equally tempered intervals} used in this score are \textbf{semi tones}, \textbf{quarter tones}, and \textbf{eighth tones}. An \textbf{inverted flat symbol} indicates a quarter tone flat, and a \textbf{sharp symbol with one vertical line} indicates a quarter tone sharp. Any accidental may be altered with an \textbf{arrow pointing upward} to indicate an eighth tone sharp, or an \textbf{arrow pointing downward} to indicate an eighth tone flat. \textbf{\circled{2} Justly tuned intervals} are indicated by the use of \textbf{Helmholtz-Ellis accidental system} combined with \textbf{cent deviations from equal temperament} for use with an electronic tuner. When no example pitch is given with the cent deviation, the mark is a deviation of the nearest “standard” accidental. In the absence of electronic tuners, approximations of these deviations are acceptable. When Helmholtz-Ellis notation is not given, the pitches are to be played as usual. The accidentals for Justly-intoned pitches are always present before the note head.
\endgroup

\end{document}
